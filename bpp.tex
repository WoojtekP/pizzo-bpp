\documentclass[compress]{beamer}

\mode<presentation> {

\usetheme{Warsaw}

\setbeamertemplate{footline}

\setbeamertemplate{navigation symbols}{}
}

\usepackage{polski}
\usepackage[utf8]{inputenc}
\usepackage{verbatimbox}
\usepackage{graphicx}
\usepackage{booktabs}
\usepackage{ragged2e}
\usepackage{scrextend}
\changefontsizes{14pt}

\title{BPP}

\begin{document}

\begin{frame}
\titlepage
\end{frame}

\begin{frame}
\frametitle{Definicja}
\justifying
\(L \in BPP \Leftrightarrow \) istnieje PMT \(M\) taka, że:
\begin{itemize}
	\item \(M\) działa w czasie wielomianowym
	\item \(\forall x. ~ x \in L ~ Pr[M(x) = 1] \geq \frac{2}{3} \)
	\item \(\forall x. ~ x \notin L ~ Pr[M(x) = 1] \leq \frac{1}{3} \)
\end{itemize}
\end{frame}

\begin{frame}
\justifying
Klasa \textbf{BPP} to klasa języków rozpoznawalnych przez probabilistyczne maszyny Turinga działające w czasie wielomianowym z prawdopodobieństwem błędu \(\frac{1}{3}\).
\end{frame}

\begin{frame}
\justifying
Klasa \textbf{BPP} to klasa języków rozpoznawalnych przez probabilistyczne maszyny Turinga działające w czasie wielomianowym z prawdopodobieństwem błędu \(\frac{1}{4}\).
\end{frame}

\begin{frame}
\justifying
Klasa \textbf{BPP} to klasa języków rozpoznawalnych przez probabilistyczne maszyny Turinga działające w czasie wielomianowym z prawdopodobieństwem błędu \(\epsilon < \frac{1}{2} \).
\end{frame}

\begin{frame}
\frametitle{Lemat o wzmacnianiu}
\justifying
Niech \(\epsilon\) będzie wartością z przedziału \((0, \frac{1}{2})\). Wówczas dla dowolnego wielomianu \(p(n)\) i PMT \(M_1\) działającej w czasie wielomianowym z błędem \(\epsilon\) istnieje równoważna jej PMT \(M_2\) działająca w czasie wielomianowym z błędem \(2^{-p(n)}\)
\end{frame}

\begin{frame}
	\frametitle{Lemat o wzmacnianiu: dowód}
	\(M_2=\) "Dla słowa \(w\):
	\begin{itemize}
		\item Oblicz wartość \(k\) (wyjaśnienie później)
		\item Wykonaj \(2k\) niezależnych symulacji maszyny \(M_1\)
		\item Jeśli większość wyników jest akceptująca, to zaakceptuj; w przeciwnym przypadku odrzuć"
	\end{itemize}
\end{frame}

\begin{frame}
\frametitle{Otwarte problemy}
\begin{itemize}
	\item Czy \textbf{BPP = P}?
	\item Jaka jest relacja między \textbf{BPP} a \textbf{NP}?
\end{itemize}
\end{frame}

\begin{frame}
\frametitle{Klasa RP}
\justifying
Klasa \textbf{RP} to klasa języków rozpoznawalnych przez PMT działające w czasie wielomianowym, które:
\begin{itemize}
	\item nie dają fałszywych odpowiedzi pozytywnych
	\item z prawdopodobieństwem co najwyżej \(\frac{1}{2}\) udzielają fałszywej odpowiedzi negatywnej.
\end{itemize}
\end{frame}

\begin{frame}
\frametitle{Klasa RP}
\justifying
\(L \in RP \Leftrightarrow \) istnieje PMT \(M\) taka, że:
\begin{itemize}
	\item \(M\) działa w czasie wielomianowym
	\item \(\forall x. ~ x \in L ~ Pr[M(x) = 1] \geq \frac{1}{2} \)
	\item \(\forall x. ~ x \notin L ~ Pr[M(x) = 1] = 0 \)
\end{itemize}
\end{frame}

\begin{frame}
\frametitle{Klasa ZPP}
\[\text{ZPP = RP} \cap \text{co-RP}\]
Jest to klasa języków, dla których probabilistyczne maszyny Turinga nie popełniają błędu.
\end{frame}

\begin{frame}
\frametitle{Algorytmy probabilistyczne}
\begin{table}
	\begin{tabular}{l | c | c}
	& Czas działania & Poprawność\\
	\hline \hline
	Lax Vegas (ZPP) & probabilistyczny & pewna \\
	\hline
	Monte Carlo (RP) & pewny & probabilistyczna
	\end{tabular}
\end{table}
\end{frame}

\begin{frame}
\frametitle{Kilka faktów:}
\begin{itemize}
	\item klasa \textbf{BPP} jest zamknięta na dopełnienie
	\item \(ZPP \subseteq RP \subseteq BPP \subseteq PP\)
	\item \(BPP \subseteq PSPACE\)
	\item jeśli \(NP \subseteq BPP\) to \(NP=RP\)
\end{itemize}
\end{frame}

\end{document} 